\documentclass[a4 paper, 12pt]{article}
\usepackage[brazilian]{babel}
\usepackage[utf8]{inputenc}
\usepackage[T1]{fontenc}
\usepackage{minted}
\usepackage{anysize}
\marginsize{3cm}{3cm}{2cm}{2cm}
\usemintedstyle{tango}
\newminted{cpp}{linenos, mathescape, frame=topline, numberblanklines=false}
\newmint{cpp}{frame=single}
\usepackage{algorithmic}
\usepackage[brazilian]{algorithm}

\renewcommand{\algorithmicrequire}{\textbf{Entrada:}}
\renewcommand{\algorithmicensure}{\textbf{Saída:}}
\renewcommand{\algorithmicend}{\textbf{fim}}
\renewcommand{\algorithmicif}{\textbf{se}}
\renewcommand{\algorithmicthen}{\textbf{então}}
\renewcommand{\algorithmicelse}{\textbf{else}}
\renewcommand{\algorithmicelsif}{\algorithmicelse\ \algorithmicif}
\renewcommand{\algorithmicendif}{\algorithmicend\ \algorithmicif}
\renewcommand{\algorithmicfor}{\textbf{para}}
\renewcommand{\algorithmicforall}{\textbf{para todos}}
\renewcommand{\algorithmicdo}{\textbf{faça}}
\renewcommand{\algorithmicendfor}{\algorithmicend}
\renewcommand{\algorithmicwhile}{\textbf{enquanto}}
\renewcommand{\algorithmicendwhile}{\algorithmicend}
\renewcommand{\algorithmicloop}{\textbf{loop}}
\renewcommand{\algorithmicendloop}{\algorithmicend\ \algorithmicloop}
\renewcommand{\algorithmicrepeat}{\textbf{repita}}
\renewcommand{\algorithmicuntil}{\textbf{enquanto}}
\renewcommand{\algorithmicprint}{\textbf{imprima}}
\renewcommand{\algorithmicreturn}{\textbf{retorne}}
\renewcommand{\algorithmictrue}{\textbf{verdadeiro}}
\renewcommand{\algorithmicfalse}{\textbf{falso}}


\renewcommand{\theFancyVerbLine}{
  \sffamily\textcolor[rgb]{0.5,0.5,0.5}{\scriptsize\arabic{FancyVerbLine}}}

\title{Projeto da Linguagem de Programa\c c\~ao - Fase 1}
\author{Alexandre Garcia, Mairieli Wessel, Felipe Fronchatti}
\date{}

\begin{document}
\maketitle
\tableofcontents
\section{Fase 1}
\subsection{Dom\'inio}
\subsubsection{Introdu\c c\~ao}

\subsubsection{Descri\c c\~ao do dom\'inio}

\subsection{Proposta da linguagem de programa\c c\~ao}

\subsection{Elementos essencias \`a linguagem}

\section{Fase 2}
\subsection{Tipos de dados}
\subsubsection{Tipos primitivos}
\subsubsection{Tipos compostos}
\subsection{Express\~oes}
\subsubsection{Literais}

\subsection{Fun\c c\~oes}

\subsection{Ordem de avalia\c c\~ao}
\subsection{Comandos}
\subsubsection{Declara\c c\~ao}
\subsubsection{Atribui\c c\~ao}

\subsection{Estruturas de controle de fluxo}
\subsubsection{Condicionais}

\subsubsection{Repeti\c c\~ao}
\subsection{Vincula\c c\~ao}
\subsubsection{Forma de vincula\c c\~ao}

\subsubsection{Tempo de vincula\c c\~ao}
\subsection{Sistema de tipos}


\end{document}